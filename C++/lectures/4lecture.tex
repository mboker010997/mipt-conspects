\section{4-я лекция}

\subsection{Понятия l-value, r-value}
Сейчас введём их неформально: l-value - то, что может стоять слева от оператора присваивания, r-value - то, что не может.

\begin{Verbatim}[tabsize=4]
	int x = 1; \\ OK
	x + 5 = 10; \\ Wrong
	10 = x; \\ Wrong
	++x = 5; \\ OK
	x++ = 5; \\ Wrong
	(a = b) = c; \\ OK
	(x + 5)++; \\ Wrong
	(x = 5)++; \\ OK
	(x = x + 1) = 10; \\ OK
\end{Verbatim}

\begin{note}
	Знак равно не всегда оператор. Например:
	\begin{Verbatim}[tabsize=4]
		int x = 5;
	\end{Verbatim}
	
	Тут = - не оператор, а просто символ.
\end{note}


\subsection{Тернарный оператор}
\begin{Verbatim}[tabsize=4]
	a ? b : c
\end{Verbatim}

Если $a$ - true, то вычисляется и возвращается $b$ ($c$ даже не вычисляется). Иначе вычисляется и возвращается $c$ ($b$ даже не вычисляется).

Тернарный оператор возвращает l-value, тогда и только тогда, когда и $b$ и $c$ - l-value.

\begin{Verbatim}[tabsize=4]
	(a == 1 ? x++ : ++x) = 10; // Wrong
\end{Verbatim}

Это будет CE, независимо от того, чему равно $a$, т. к. ++x - r-value.


\subsection{Оператор <<запятая>>}

\begin{Verbatim}[tabsize=4]
	x = 1, y = 3, ++z;
\end{Verbatim}

Оператор <<запятая>> сначала высчитывает левый операнд, потом высчитывает правый операнд и возвращает то, что вернул правый операнд.

\begin{Verbatim}[tabsize=4]
	(x = 5, ++x) = 1; \\ OK
\end{Verbatim}

Однако <<запятая>> не всегда оператор.

\begin{Verbatim}[tabsize=4]
	int x = 2, y = 1; \\ Not operator
	f(x, y); \\ Not operator
\end{Verbatim}

Приоритет оператора <<запятая>> самый маленький из всех.

\begin{Verbatim}[tabsize=4]
	int x = 1, 5; \\ x = 1, but return value - 5;
\end{Verbatim}


\subsection{Subscription (обращение по индексу)}

\begin{Verbatim}[tabsize=4]
	a[5]; \\ l-value
\end{Verbatim}

Оператор <<квадратные скобки>> - обращение по индексу (возвращает l-value).


\subsection{Sizeof}

Sizeof(x) - это оператор, возвращающий кол-во байт, которое занимает переменная $x$. При этом компилятор не вычисляет само значение $x$.

\begin{Verbatim}[tabsize=4]
	cout << sizeof(++x); \\ 4
\end{Verbatim} 

От этой команды $x$ не поменяется. Sizeof() вычислят кол-во байт, которое занимала бы переменная, будь она посчитанная.


\subsection{Приоритет операторов}

\href{https://en.cppreference.com/w/cpp/language/operator_precedence}{Таблица приоритетов}


\subsection{Ошибка компиляции (CE)}

Бывает трёх видов: лексические, синтаксические и семантические.










