\section{5-я лекция}

\subsection{Runtime Error (RE)}

\begin{Verbatim}[tabsize=4]
int a = 5 / 0;
\end{Verbatim}

Целочисленное деление на ноль - это RE.

Segmentation fault (Segfault) - ошибка сегментирования (обращение к элементу массива, которого не было, и не только) - тоже RE. Это ошибка на уровне системы, не на уровне языка (также как Floating point exception).

\subsection{Undefined Behaviour}

\begin{itemize}
	\item Обращение за границу массива (segfault - частный случай UB)
	\item Переполнение int (классный пример представлен в \href{https://youtu.be/swyvjLAV9-k?t=1461}{лекции})
\end{itemize}

\subsection{Unspecified Behaviour}
Подробнее в \href{https://youtu.be/swyvjLAV9-k?t=1702}{лекции}

\begin{Verbatim}[tabsize=4]
	int f() {
		cout << 1;
		return 1;
	}
	
	int g() {
		cout << 2;
		return 2;
	}
	
	int h() {
		cout << 3;
		return 3;
	}
	
	int main() {
		cout << f() * g() + h(); \\ ***5
	}
\end{Verbatim}

В конце точно выведется 5, а что будет до этого - неизвестно (f, g, h могут считаться в каком угодно порядке). Порядок вычислений не то же самое, что приоритет операторов.


\subsection{Указатели}

У каждой переменной есть свой адрес (\&a - получить адрес переменной a). Возвращаемый тип этой операции - $T^{*}$, где $T$ - тип переменной. Операнд к \& должен быть l-value. Указатель - переменная, хранящая адрес другой переменной (тип $T^{*}$). Можно взять адрес указателя, т. к. указатель - тоже переменная.

\begin{Verbatim}[tabsize=4]
	int** p = &(&a); \\ Wrong (&a - r-value)
\end{Verbatim}

Обратная операция - * (по адресу вернуть значение, хранящееся по этому адресу). *: $T^{*} \rightarrow T$.

\begin{Verbatim}[tabsize=4]
	int* a, b;
\end{Verbatim}

В итоге, $a$ - это указатель, $b$ - нет. Поэтому не рекомендуется объявлять несколько переменных на одной строке.

\begin{Verbatim}[tabsize=4]
	int a = 1;
	int* p = &a;
	{
		int b = 2;
		p = &b;
	}
	cout << *p; \\ UB
\end{Verbatim}

\subsection{Реализация swap}

\begin{Verbatim}[tabsize=4]
	void swap(int* a, int* b) {
		int t = *a;
		*a = *b;
		*b = t;
	}
\end{Verbatim}


\subsection{Операции над указателями}

\begin{itemize}
	\item Инкремент - перейти в следующую ячейку памяти, сдвинутую на sizeof(T) байт
	\item Декремент
	\item Сложение с int - сделать инкремент n раз
	\item Разность указателей - сколько между ними шагов размера sizeof(T).
\end{itemize}









