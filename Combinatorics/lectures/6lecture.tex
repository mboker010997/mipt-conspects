\section{Порядки}

\begin{definition}
	Порядок плотен, если для $(\forall x, y)\ ( (x < y) \rightarrow (\exists z \ | \ x < z < y) )$
\end{definition}

\begin{theorem}{Теорема о плотном порядке}
	Любые два счетных плотно линейно упорядоченных множества без наименьшего и наибольшего элементов изоморфны.
\end{theorem}
\begin{proof}
	$A = \{a_1, a_2, \ldots\}, B = \{b_1, b_2, \ldots\}$.
	Чтобы получился изоморфизм, нужно каждый раз брать подходящее b с наименьшим номером.
\end{proof}