\section{Наивная теория множеств}

\begin{definition}
	Множество - это совокупность некоторых объектов. Каждый объект входит в множество не более 1 раза, иначе - это мультимножество.
\end{definition}

\begin{definition}
	$x \in Y$. Объект $x$ является элементом множества $Y$.
\end{definition}

\begin{definition}
	$X \subset Y$. Множество $X$ является подмножеством множества $Y$. $X \subset Y \stackrel{def}{\Leftrightarrow} (a \in X \Rightarrow a \in Y)$
\end{definition}

\begin{definition}
	Множества $X$ и $Y$ равны, если $(X \subset Y) \land (Y \subset X)$.
\end{definition}

\begin{proposition}
	Равенство множеств обладает следующими свойствами:
	\begin{enumerate}
		\item Рефлексивность $(X = X)$.
		\item Транзитивность $((X = Y) \land (Y = Z) \Rightarrow X = Z)$.
		\item Симметричность $(X = Y \Rightarrow Y = X)$.
	\end{enumerate}
\end{proposition}

\begin{proposition}
	Отношение подмножества обладает следующими свойствами:
	\begin{enumerate}
		\item Рефлексивность $(X \subset X)$.
		\item Транзитивность $((X \subset Y) \land (Y \subset Z) \Rightarrow X \subset Z)$.
		\item Антисимметричность $((X \subset Y) \land (Y \subset X) \Rightarrow Y = X)$.
	\end{enumerate}
\end{proposition}

\subsection{Операции над множествами}
\begin{enumerate}
	\item Объединение $A \cup B = \{x \ | \ (x \in A) \lor (x \in B)\}$
	\item Пересечение $A \cap B = \{x \ | \ (x \in A) \land (x \in B)\}$
	\item Разность $A \setminus B = \{x \ | \ (x \in A) \land (x \notin B)\}$
	\item Симметрическая разность $A \triangle B = \{x \ | \ (x \in A) \oplus (x \in B)\}$
	\item Дополнение $\overline{A} = \{x \ | \ x \notin A\}$
\end{enumerate}

\subsection{Упорядоченные пары и кортежи}
Законы де Моргана:
\[\overline{A \cap B} = \overline{A} \cup \overline{B}\] \[\overline{A \cup B} = \overline{A} \cap \overline{B}\]

\begin{definition}
	Неупорядоченная пара - множество из двух элементов (возможно, одинаковых). Обозначение: $\{a, b\}$.
\end{definition}

\begin{definition}
	Упорядоченная пара - неупорядоченная пара, в которой зафиксирован первый элемент. Обозначение: $(a, b)$.
\end{definition}

\begin{definition}
	Упрощенное определение Куратовского: $(a, b) = \{\{a, b\}, a\}$.
\end{definition}

\begin{definition}
	Полное определение Куратовского: $(a, b) = \{\{a, b\}, \{a\}\}$.
\end{definition}

\begin{definition}
	Декартово произведение $A \times B = \{(a, b) \ | \ (a \in A) \land (b \in B)\}$.
\end{definition}

\begin{definition}
	Кортеж длины 0 - это $\varnothing$. Кортеж длины $(n + 1)$ - это $\{a, \{a, t\}\}$, где $t$ - кортеж длины $n$.
\end{definition}